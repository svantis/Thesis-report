% USE PDFLatex!
% to correctly render Swedish characters

\documentclass{popsci}

\usepackage[utf8]{inputenc}
\usepackage[swedish, english]{babel}

\usepackage{fancyhdr}
\usepackage{titling}
\usepackage{color}
\usepackage{colortbl}
\usepackage{graphicx}
\usepackage{flushend}
\usepackage{lmodern}


% Please specify the presentation date
\presentationsdag{2021-02-12}

% use either of these commands to specify the title of your thesis
\examensarbete{Emotion Recognition in Written Conversations Using CASTLE}
% To create a title in two rows, leave examensarbete blank and fill in examensarbeteTwoRows.
%\examensarbeteTwoRows{Application Specific Instruction-set Processor Using a Parametrizable multi-SIMD}{Synthesizeable Model Supporting Design Space Exploration}
\student{Nick Persson, Filip Karabeleski}
%\students{Magnus Hultin}{Mr X}
\supervisor{Pierre Nugues (LTH)}
\examiner{Flavius Gruian (LTH)}

% Your pop-sci title should be different (more catchy) than your thesis title
\title{Identifiering av känslor i skriftliga konversationer}


\begin{document}

% not more than 4 rows!
\theabstract{Identifiering av känslor i konversationer är ett notoriskt svårt problem för datorer att förstå då konversationer ofta innehåller subtila indikationer som kan vara svåra att urskilja, till och med för en människa. Vi har undersökt möjligheten att förbättra datorers kompetens inom detta område.


}
\ifx false
\begin{figure}[!bth] % Use pictures in your pop-vet!
\includegraphics[width=\linewidth]{msccls/popsci/popsci_image.png} 
%\caption{En fin bild}
\end{figure}
\fi

{\noindent 

Identifiering av känslor är en form av kategorisering där utvecklarna sammanställer en lista av känslor att använda som kategorier. Därefter går datorn igenom en samling data, eller dataset, och kategoriserar varje mening med den känslan som passar bäst av de känslor i listan. 


Detta görs oftast med hjälp av maskininlärning, då det hade varit alldeles för tidskrävande att göra det manuellt. Maskininlärning kan låta som magi, men är i själva verket bara en upprepande process där svar på uppgiften matas in till datorn och den lär sig mönster och hur den ska hitta svar på liknande uppgifter. Den här processen kallas träning.  Ett vanligt problem med maskininlärning är att datorn kan få svårigheter att förutse nya sorters uppgifter som skiljer sig från det den tidigare lärt sig av. 



%För att öka prestandan i dagens processorer finns det vektorenheter och flera kärnor i processorer. Vektorenheten finns till för att kunna utföra en operation på en mängd data samtidigt och flera kärnor gör att man kan utföra fler instruktioner samtidigt. Ofta är processorerna designade för att kunna stödja en mängd olika datorprogram. Detta resulterar i att det blir kompromisser som kan påverka prestandan för vissa program och vara överflödigt för andra. I t.ex. videokameror, mobiltelefoner, medicinsk utrustning, digital kameror och annan inbyggd elektronik, kan man istället använda en processor som saknar vissa funktioner men som istället är mer energieffektiv. Man kan jämföra det med att frakta ett paket med en stor lastbil istället för att använda en mindre bil där samma paketet också skulle få plats.

I vårt examensarbete har vi förbättrat en av de bäst presterande modellerna som finns i dagsläget. Vi har ändrat den inre arkitekturen men bibehållit dess logik och beteende. Vad detta leder till är en förbättrad modell som vi har döpt till CASTLE. Denna "tänker" på samma sätt som tidigare men utför mer sofistikerade beräkningar för att få fram ett mer precist svar. CASTLE har testats på några av de mest förekommande dataseten inom området. Dataseten är strukturerade som dialoger mellan flertalet personer och kan vara allt ifrån improvisation till direkta utdrag ur den populära 90-tals serien Vänner. 

Att testa på samma dataset som de andra gör ger mer tillförlitlighet till resultaten. 




%I mitt examensarbete har jag skrivit en modell som kan användas för att snabbt designa en processor enligt vissa parametrar. Dessa parametrar väljs utifrån vilket eller vilka program man tänkta köra på den. Vissa program kan t.ex. lättare använda flera kärnor och vissa program kan använda korta eller längre vektorenheter för dess data.

Modellen testades med olika multimedia program. Den mest beräkningsintensiva och mest upprepande delen av programmen användes. Dessa kallas för kärnor av programmen. Kärnorna som användes var ifrån MPEG och JPEG, som används för bildkomprimering och videokomprimering.


Resultatet visar att det finns en prestanda vinst jämfört med generella processorer men att detta också ökar resurserna som behövs. Detta trots att den generella processorn har nästan dubbelt så hög klockfrekvens än dem applikations-specifika processorerna. Resultatet visar också att schemaläggning av instruktionerna i programmen spelar en stor roll för att kunna utnyttja resurserna som finns tillgängliga och därmed öka prestandan. Med den schemaläggningen som utnyttjade resurserna bäst var prestandan minst 79\% bättre än den generella processorn.
}

\end{document}
